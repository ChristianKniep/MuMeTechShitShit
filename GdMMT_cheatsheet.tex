\documentclass[10pt,landscape]{article}
\usepackage{multicol}
\usepackage{calc}
\usepackage{ifthen}
\usepackage[landscape]{geometry}
\usepackage[ngerman]{babel}
\usepackage[utf8]{inputenc}
\usepackage{graphics}
\usepackage{listings}
\usepackage{verbatim}


% This sets page margins to .5 inch if using letter paper, and to 1cm
% if using A4 paper. (This probably isn't strictly necessary.)
% If using another size paper, use default 1cm margins.
\ifthenelse{\lengthtest { \paperwidth = 11in}}
	{ \geometry{top=.5in,left=.3in,right=.5in,bottom=.5in} }
	{\ifthenelse{ \lengthtest{ \paperwidth = 297mm}}
		{\geometry{top=1cm,left=1cm,right=1cm,bottom=1cm} }
		{\geometry{top=1cm,left=1cm,right=1cm,bottom=1cm} }
	}

% Turn off header and footer
\pagestyle{empty}
 

% Redefine section commands to use less space
\makeatletter
\renewcommand{\section}{\@startsection{section}{1}{0mm}%
                                {-1ex plus -.5ex minus -.2ex}%
                                {0.5ex plus .2ex}%x
                                {\normalfont\large\bfseries}}
\renewcommand{\subsection}{\@startsection{subsection}{2}{0mm}%
                                {-1explus -.5ex minus -.2ex}%
                                {0.5ex plus .2ex}%
                                {\normalfont\normalsize\bfseries}}
\renewcommand{\subsubsection}{\@startsection{subsubsection}{3}{0mm}%
                                {-1ex plus -.5ex minus -.2ex}%
                                {1ex plus .2ex}%
                                {\normalfont\small\bfseries}}
% Special commands
\newcommand{\code}[1]{\texttt{#1}}

\makeatother

% Define BibTeX command
\def\BibTeX{{\rm B\kern-.05em{\sc i\kern-.025em b}\kern-.08em
    T\kern-.1667em\lower.7ex\hbox{E}\kern-.125emX}}

% Don't print section numbers
\setcounter{secnumdepth}{0}


\setlength{\parindent}{0pt}
\setlength{\parskip}{0pt plus 0.5ex}


% -----------------------------------------------------------------------

\begin{document}

\raggedright
\footnotesize
\begin{multicols}{3}


% multicol parameters
% These lengths are set only within the two main columns
%\setlength{\columnseprule}{0.25pt}
\setlength{\premulticols}{1pt}
\setlength{\postmulticols}{1pt}
\setlength{\multicolsep}{1pt}
\setlength{\columnsep}{2pt}

\begin{center}
     \Large{\textbf{MuMeTech-CheatSheet}} \\
\end{center}

\section{Definition}
\subsection{MuMeTech} ist rechnergef\"uhrt, unabh\"angig, diskret und kontinuierlich.\\
\subsection{Kompression}
Daten/Datenkan\"ale werden auf bestimmte Aufl\"osung/Genauigkeit/Abtastrate reduziert (Bei unterschiedlicher Reduzierung je nach Kanal, nennt man es Subsampling).
\subsection{\"Ubertragungsmodi}
\begin{itemize}
    \item \textbf{synchron} Der Sender sendet direkt an den Empf\"anger, es kann erst weitergesendert werden, wenn die Daten empfangen werden. (Handy)
    \item \textbf{asynchron} Die Daten werden w\"ahrend der \"Ubertragung zwischengepuffert, womit der Sender nicht auf den Empf\"anger warten muss. (Post,EMail)
    \item \textbf{isochron} Zeitraster ist fest, konstante Periode und Datenrate. (USB)
\end{itemize}

\subsection{Medienarten}
\begin{itemize}
    \item \textbf{Perzeptionsm.} Wahrnehmung
    \item \textbf{Repr\"asentationsm.} Darstellung
    \item \textbf{Pr\"asentationsmedium} Ausgabe
    \item \textbf{Speichermedium} Physikalischer Datenspeicher
\end{itemize}

\section{Kompressionsarten}
\subsection{Huffmann}
Zeichen werden nach ihrer H\"aufigkeit geordnet mit verschieden langen Codes repr\"asentiert.
\subsection{Laufl\"angenkodierung}
Fasst direkt aufeinanderfolgende Zeichenketten zusammen. ($aaabbb \Rightarrow 3a3b$)

\section{Erdn\"usse}
\subsection{FFT}
Spaltet komplexes Signal in mehrere reine Sinusschwingungen auf, welche addiert das Orginalsignal ergeben.
\subsection{DCT}
Wichtige Elemente der darzustellenden Daten werden mit mehr Bandbreite versehen (Links-Oben-Bild)

\section{Audio}
\subsection{Begriffe}
\begin{itemize}
    \item \textbf{Phon} Empfundene Lautst\"arke im Verh\"altnis zu 1000 Hz Sinus. Skaliert normal (nicht log.).
    \item \textbf{Dezibel} Logarithmisch ausgedr\"uckte Lautst\"arke 6dB Unterschied bedeuten Verdoppelung der Lautst\"arke.
    \item \textbf{Frequenzamplitude} Amplitude wird angegeben in Dezibel und bestimmt die Lautst\"arke. Beschreibt maximale Auslenkung der Sinuswelle.
    \item \textbf{Klang} Schallwelle die vom menschlichen Ohr als bestimmter Ton wahrgenommen wird.
\end{itemize}
\subsection{Analog2Digital}
\begin{enumerate}
    \item \textbf{Vorverarbeitung} Filterung (St\"orger\"ausche), Verst\"arkung (Dynamikausnutzung) \\
    Im zweiten Schritt erfolgt eine Frequenzbandbegrenzung (Tiefpassfilter) auf $1/2$ der Abtastfrequenz (Shannon Abtasttherorem)
    \item \textbf{Abtastung} In konstanten Intervallen wird der Wert des Eingangssignals entnommen.
    \item \textbf{Quantisierung} Diskretisierung des bei der Abtastung ermittelten Wertes 
    \item \textbf{Kodierung} Bin\"arkodierung der Signalproben
\end{enumerate}
\textbf{Zusammenfassend} \\
Aus einem zeitkontinuierlich ablaufendem VOrgang werden Signalproben genommen und in ihrer Amplitude quantisiert und in eine computergerechte Darstellung gebracht.

\subsection{Kodierungsmethoden}
\textbf{Verlustbehaftet}
\begin{itemize}
    \item \textbf{PulseCodeModulation - PCM} 3 Schritte \\
        \begin{itemize}
        \item \textbf{Schritt 1} Abtastung mit zeitlich konst. Rate
        \item \textbf{Schritt 2} Quantisierung der Werte
        \item \textbf{Schritt 3} Kodierung in bin\"arcode \\
            Die Kodierung erfolt linear. 
        \end{itemize}
    \item \textbf{DPCM} Differenzielle PCM \\
        Die Quantisierung erfolgt anhand der Differenz zu einer Vorhersage.
    \item \textbf{DeltaModulation} Eine DPCM mit nur einem Bit. Wertebereich -/+1.
    Die Sch\"atzwerte nehmen dabei immer an, dass der neue Abtastwert gleich dem vorherigem ist.
    \item \textbf{Adaptive differenzielle PCM} \"Ahnlich DPCM, jedoch mit dyamischer Vorhersage.
     Angepasste Quantisierung, dadurch bessere Quali.
\end{itemize}
\subsection{Kompressionsverfahren f\"ur Audio}
\subsubsection{Datenreduktion}
Filterung der Daten (z.B. nach psychoakustik).
\subsubsection{Datenkompression}
Verlustfreie Komprimierung der Daten.

\subsection{mp3}
\begin{enumerate}
    \item PCM (768Kbit/s)
    \item Filterbank f\"ur 32 Subb\"ander / FastFourierTrans f\"ur 1024 Abtastwerte
    \item FFT $\Rightarrow$ PsychAkModel nun wird anhand derer und der Subb\"ander quantisiert.
    \item Audiodatenkodierung mit Huffmann, Nebeninfos codiert
    \item BitstromFormatierung und Fehlerkorrektur
\end{enumerate}

\subsection{MIDI}
Daten\"ubertragungsprotokoll f\"ur Musikdaten.\"Ubertragen werden Steuerinformationen zwischen elektronischen Instrumenten, welche von Programm interpretiert werden k\"onnen. \\
Inhalt zum Beispiel: Anschlagst\"arke, Lautst\"arke, MidiKanalnummer (4Bit), Spurname \\
\begin{itemize}
    \item[Format 0] Alle Midikan\"ale sind in einer Spur zusammengefasst, somit keine gleichzeitigen Anschl\"age verschiedener Instrumente (Klingelton) 
    \item[Format 1] Jeder Kanal hat eigene Spur, somit k\"onnen auch gleichzeitige Anschl\"age realisiert werden.
    \item[Format 2] Im Format 2 besteht jede Spur (Track) aus unabh\"angigen Einheiten. Im Gegensatz zu SMF 1 k\"onnen also mehrere Spuren dieselbe MIDI-Kanal-Nummer haben.
\end{itemize}
\textbf{THRU-Port} gibt parallel zum Outport eines Ger\"ates (z.B. Filter) das unbehandelte Inputsignal aus (z.B. f\"ur Aufnahmen).

\subsection{Beispielrechnungen}
\textbf{Einheiten:} \\
$1MB \Rightarrow 10^6 Byte || 1MiB \Rightarrow 1024 \times 1024 Byte$ \\
$1GB \Rightarrow 10^9 Byte || 1GiB \Rightarrow 1024 \times 1024 \times 1024 Byte$ \\
\subsubsection{PCM: 44.1KHz,16Bit,stereo,20min}
$44.100 \times 16 \times 2 \times 20 \times 60 \Rightarrow Bit$ \\
$44.100 \times 2 \times 2 \times 20 \times 60 \Rightarrow 2116800 Byte \approx 2.1 MB$ \\
\subsubsection{MP3: 128kbit/s}


\section{Grafiken/Bilder}
\subsection{Farbmodi}
\begin{itemize}
    \item \textbf{RGB} RotGr\"unBlau.\\
        Additive Farbmischung mit drei Farbkan\"ale a 8Bit (default).
        \begin{itemize}
            \item \textbf{Anwendungen} Monitordarstellung, Kamera
            \item \textbf{Vorteile} \\
                Gut auf Ger\"aten anzuwenden, die Lichtquellen aussenden.\\
                Direkt mit Algo bearbeitbar\\
                Darstellungskapazit\"at vieler Farbnuancen
            \item \textbf{Nachteil} \\
                Probleme mit Darstellung von Schwarz \\
                Ger\"ateabh\"angig. \\
                8 \% des Farbraums sind nicht wahrnehmbare Farben \\
                Helligkeitskorrektur schwer \\
                Eignet sich nicht f\"ur Druck (Additiv/Substraktiv)
        \end{itemize}
    \item \textbf{YUV}
        Darstellung durch Luminanz (Y) und Chrominanz (UV).
        \begin{itemize}
            \item \textbf{Anwendungen} Analoges NTSC/PAL-Farbfernsehen
            \item \textbf{Vorteile} \\
                Halbe Bandbreite von RGB \\
                Durch Subsampling optimierung m\"oglich (siehe Subsampling)\\
                Vollst\"andiger Farbraum abgedeckt\\
                Abw\"artskompatibel zu Schwarz/Weiss\\
                Ausnutzung Wahrnehmungspsychologie\\
                Helligkeit separat im Gegensatz zu RGB (jeder Kanal muss angepasst werden)\\
                Progressive Vollbilder m\"oglich
            \item \textbf{Nachteil} \\
                Verteilung der Farbanteile der Cyan/Orange und Megenta/Gr\"un ist ungleichm\"assig auf U und V, daher keine Bandbreitenreduktion m\"oglich
        \end{itemize}
    \item \textbf{YIQ}
        Darstellung durch Luminanz (Y), sowie den Farbdifferenzen I (Cyan/Orange) und Q (Magenta/Gr\"un) \\
        Irgendwie zu YUV verdreht! WHY? How much?
        \begin{itemize}
            \item \textbf{Anwendungen} Altes analoges NTSC-Farbfernsehen 
            \item \textbf{Vorteile} \\
                \"Ahnlich YUV \\
                Kommt wahrscheinlich nicht in der Klausur dran (Jonas)
            \item \textbf{Nachteil} \\
                Nur \"uberm Teich im Gebrauch
        \end{itemize}
\end{itemize}

\subsection{JPEG}
\subsubsection{JPEG-Kodierung}
\begin{itemize}
    \item \textbf{Bildvorverarbeitung (verlustfrei)}
        \begin{itemize}
            \item Grauwerttransformation (Kontrasterh\"ohung und Helligkeitsverbesserung)
            \item Bildfilterung (Rauschunterdr\"uckung,Kantenverst\"arkung,Gl\"attung,...)
        \end{itemize}
    \item \textbf{Bildverarbeitung (theo. verlustfrei)}
    \begin{itemize}
        \item Abtastung und Digitaliserung der Bildinformationen.
        \item Einteilung in 8x8-Pixel-Bl\"ocke, wobei jeder Pixel mit 8bit kodiert wird (optimaler Kompromis zwischen Laufzeit und Quali; Zahl f\"ur DCT).
        \item DCT Der 8Bit-Farbwert wird vom Ortsbereich- in den Frequenzbereich transformiert. \\
            Das Ergebnis ist eine 8x8-Frequenzraummatrix S, $S_{00}$ entspricht dem Anteil der Freuqenz 0 (Grundfarbton), dieser ist der DC-Koeffizient.\\
            Alle anderen $S_{ij}$ heissen AC-Koeffizienten und geben Auskunft \"uber die Frequenzver\"anderungen (Farbver.) innerhalb des Blockes.\\
            Der letzte Eintrag $S_{77}$ gibt dabei die h\"ochste in beiden Richtungen auftretene Frequenz an.
    \end{itemize}
    \item \textbf{Quantisierung (verlustbehaftet)} Erstellen einer ZickZack-Sequenz (Diagonaler Schnitt von links-oben an). Ausnutzung des PsychoVisuellenModels (PVM).
        Die Anwendung stellt eine Liste mit 64 Faktoren zur Verf\"ugung. Anhand dieser werden die DCT-Koeffizienten gewichtet (und gerundet),
        wodurch die Freuqenzwechsel an die Qualit\"atsanforderung angepasst werden.\\
        Wird die Qualit\"at reduziert, so ist die rechte untere Dreiecksmatrix mit Nullen versehen. Dies kommt der Entropiekodierung zu Gute.
    \item \textbf{Entropiekodierung (verlustfrei)} Die resultierdende Liste wird mit Huffmann oder aritmetisch kodiert.    
\end{itemize}
\subsubsection{JPEG-Modi}
\begin{itemize}
    \item \textbf{Sequenzielle mode} Das Bild wir din einem einzigen Durchlauf kodiert.
    \item \textbf{progressive mode} Das Bild wird in mehreren Durchl\"aufen immer genauer kodiert.\\
        Vorteil: Schnelle (grobpixelige) Vorschau des Bildes
    \item \textbf{Hirachischer Modus} Das Bild wird in verschiedenen Aufl\"osungen kodiert.\\
        Vorteil: Jede Anwendung greift sich ihre geeignete Aufl\"osung heraus und muss nicht rekodieren.
    \item \textbf{lossless mode} Verlustfreie Kodierung des Bildes
\end{itemize}

\section{Netzwerk/Internet}
\subsection{IP-Adresse/Subnetzmaske}
\subsection{\"Ubertragungsarten}
\subsection{AJAX}
\subsection{HTML5}
\subsection{HTTP}

\section{MPEG}
\subsection{MPEG1}
\subsection{MPEG4}
\subsection{Beispielrechnung}
\subsubsection{MPEG}
Bei 800x600, 24Bit, 25fps, 60s sind es pro Minute:\\
$800 \times 600 \times 24 \times 25 \times 60 \Rightarrow 4320 \times 10^6 Byte$
\section{CD/DVD}


\end{multicols}
\end{document}
